\pagestyle{fancy}
\fancyhead[L]{\ungaramond\small\textbf{}}
\fancyhead[C]{}
\fancyhead[R]{\ungaramond\small\textbf{}}
\fancyfoot[L]{\ungaramond\small\textbf{Ridha Bouazizi}}
\fancyfoot[C]{}
\fancyfoot[R]{\ungaramond\small\textbf{Page \thepage\ sur\ \pageref{LastPage}}}
\renewcommand{\headrulewidth}{1.7pt}
\renewcommand{\headrule}{\hbox to\headwidth{\color{gray}\leaders\hrule height \headrulewidth\hfill}}
\renewcommand{\footrulewidth}{1.7pt}
\renewcommand{\footrule}{\hbox to\headwidth{\color{gray}\leaders\hrule height \footrulewidth\hfill}}
\pagenumbering{arabic}
\phantomsection\addcontentsline{toc}{chapter}{Introduction Générale} % inclure dans TdM
\begin{center}
\ungaramond{\textbf{\LARGE{Introduction Générale}}}
\fancyhead[R]{\ungaramond\small\textbf{Introduction Générale}}

\end{center}
\vspace{0.8cm}

\noindent Aujourd'hui, le monde évolue constamment et son développement dans divers domaines est assez important, voire fulgurant, en particulier dans notre domaine de la technologie informatique, qui s'est énormément développé au cours des années précédentes.
\\
La digitalisation des processus d'activité constitue actuellement une priorité stratégique des entreprises qui cherchent à mieux répondre aux besoins et attentes du client d’où l’augmentation de leur profit.

\\Dans ce cadre, la multinationale L'Oréal est parmi les grands groupes à investir en recherche et innovation notamment en matière d'optimisation et de digitalisation des processus de production de produits cosmétiques.

\\Le projet s’insère dans le cadre du développement des produits de teinte des cheveux de la marque L’Oréal. Il consiste à concevoir une application digitale permettant un alignement entre le chimiste et le marketeur en matière de choix et de validation des teintes de couleur de cheveux à mettre en production.

Le présent rapport, qui expose le travail réalisé dans ce projet, est reparti en cinq chapitres

\\Le premier chapitre présentera le cadre général du projet, l’entreprise d’accueil et la problématique posée. Ensuite, l’étude et la critique de l’existant, la présentation du projet et l’explication du choix de la méthodologie de développement.

\\Le deuxième chapitre va traiter l’exigence et la spécification des besoins fonctionnels et non fonctionnels pour définir les objectifs de ce projet ainsi que l’identification des acteurs. Ces besoins seront illustrés par le diagramme de cas d’utilisation général et par le backlog produit. On va aussi mettre l’accent sur les outils et approches utilisés afinde réaliser et mettre en œuvre ce projet commençant par la planification des sprints, l’architecture logique et physique de notre application, le diagramme de classe d’analyse, les technologies et environnement logiciel utilisés.

\\Durant le troisième chapitre nous parlerons du moteur graphique responsablede la génération des images synthétisées des teintes de cheveux, la conversion de ce dernier en client léger et enfin la création du protocole de communication entre le moteur graphique et la partie backend

\newpage
\\Dans  le  quatrième  chapitre  nous  attaquerons  la  partie  backend  où  on  va  assurer  la  communication  avec  le  moteur  graphique  pour  le  transfert  des  images  synthétisées.  On  va  aussi  préparer  les  apis  fonctionnels  responsable  à  la  gestion  des  simulations.    Ce  chapitre  est  divisé  en  trois  parties,  la  première  partie  est  phase  d’analyse  où  on  trouve  les  diagrammes  de  cas  d’utilisation  et  séquence  système  ainsi  que  le  sprint  backlog,  la  deuxième  partie  est  phase  de  conception  où  on  trouve  les  deux  diagrammes  séquence  objet  et  classe  de  conception,  et  la  troisième  est  la  phase  de  réalisation  du  sprint  où  on  trouve  les  captures  sur  les  différentes  fonctionnalités  du  sprint.

\\Dans  le  dernier  des  chapitres,  nous  présenterons  la  partie  frontend  et  l’intégration  de  notre  solution  dans  la  Formulation  Center  la  plateforme  qui  contient  toutes  les  solutions  du  département  recherche  et  développement.
Ce  chapitre  est  aussi  divisé  en  trois  parties,  la  première  partie  est  la  phase  d’analyse  où  on  trouve  les  diagrammes  de  cas  d’utilisation  et  séquence  système  ainsi  que  le  sprint  backlog,  la  deuxième  partie  est  la  phase  réalisation  où  on  trouve  les  captures  sur  les  fonctionnalités  du  sprint.
On  finira  avec  une  conclusion  générale  et  les  perspectives  de  ce  projet.
