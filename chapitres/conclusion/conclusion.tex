\phantomsection\addcontentsline{toc}{chapter}{Conclusion Générale et Perspectives}\textcolor{white}{I} %\textcolor{white}{I} inclure\textcolor{white}{I} dans\textcolor{white}{I} TdM

\begin{center}
\ungaramond{\textbf{\LARGE{Conclusion\textcolor{white}{J}Générale\textcolor{white}{J}et\textcolor{white}{J}Perspectives}}}
\fancyhead[R]{\ungaramond\small\textbf{Conclusion\textcolor{white}{J}Générale\textcolor{white}{J}et\textcolor{white}{J}Perspectives}}

\end{center}
Ce stage de fin d'études s'insère dans le cadre de la politique d'innovation du groupe L'Oréal qui place le digital au service de ses consommateurs et de ses équipes.

Pour mener ce projet, j’ai consacré une partie du stage a comprendre les concepts et les processus liés au projet. Les discussions avec l’encadrant professionnels et les lignes métier m’ont permis de cerner le besoin et la piste de développement du projet.

Le projet vise a faciliter le développement des teintes de cheveux en digitalisant le processus d'alignement entre les lignes métier (Chimie/Marketing).

Cette solution offre des nouvelles fonctionnalités permettant un meilleur choix des teintes et une optimisation du temps de validation.

En termes de perspectives, le travail d’optimisation et de digitalisation peut être poursuivi notamment par l’amélioration de la qualité des prototypes de teintes de cheveux (possibilité de passer d’un prototype a quatre vues actuellement à cinq vues, voire plus/possibilité de passer à un prototype en mode 3D).

