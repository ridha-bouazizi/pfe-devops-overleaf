\graphicspath{{./assets/}}
\setcounter{mtc}{4}
\chapter{2nd Sprint: Information gathering and cloud design }
\fancyhead[R]{\ungaramond\small\textbf{Chapter IV. 2nd Sprint: Information gathering and cloud design }}
\minitoc
\newpage

\section*{Introduction}

\section{Sprint backlog :}

\begin{longtable}[H]{|m{1.5cm}|m{3cm}|m{1.5cm}|m{8cm}|}
\hline
{\textbf{Epic ID}} & {\textbf{Epic}} & {\textbf{Story ID}} & {\textbf{Story}}\\
\hline
1  & Information gathering.	 &  1.1	 &  Research provider specific (OVH) constraints\\
\cline{3-4}
& & 1.2 & Container orchestration choice. \\
\cline{3-4}
& & 1.3	& Deciding on virtualized networking. \\
\cline{3-4}
& & 1.4	& Deciding on application load balancer. \\
\cline{3-4}
& & 1.5	& Deciding on network level load balancer. \\
\cline{3-4}
& & 1.6	& Storage backend choice.\\
\hline
\caption{Table 2nd sprint backlog}
\end{longtable}


\section{Package diagram of cloud resources :}

The cluster resources are basically cloud instances with different specifications tailored to their use cases. Three major groups of resources are distinguished in the following diagram: Control plane instances, Compute instances and storage assets which are object store buckets and compute instances to which raw block stores are attached.

\begin{figure}[H]\centering
\includegraphics[width=1.0\textwidth,angle=00]{assets/f10.png}
\caption{Package diagram of cloud resources }
\label{fig:f10}
\end{figure}

\newpage
Comparative study on container orchestration: 

Container Orchestration Engines such as “Kubernetes”, “Apache mesos” and “docker swarm” are platforms for managing containers and automating the deployment, scaling, and operations of containers across a cluster of nodes. This is achieved by pooling the discrete cloud resources into a single PaaS on which deployments can be deployed. 

\begin{longtable}[H]{|m{3.5cm}|m{3.5cm}|m{3.5cm}|m{3.5cm}|}
\hline
Criteria & Kubernetes & Docker swarm & Apache Mesos  \\
\hline
Ease of use & Medium & Easy & Complex  \\
\hline
Cluster scalability & Medium to Large & Small to Medium & Very Large  \\
\hline
Cluster installation & Complex & Easy & Medium  \\
\hline
Container deployment & YAML based  & Docker based & JSON based \\
\hline
\caption{Table Sprint planification}
\end{longtable}

Kubernetes is our container orchestration tool of choice in this project. We envision to take on the challenge of putting in place a complete sustainable PaaS by leveraging its adaptability through the use of custom resource definitions. 


\section{Package diagram for the PaaS logical components:}

\begin{figure}[H]\centering
\includegraphics[width=1.0\textwidth,angle=00]{assets/f11.jpg}
\caption{Package diagram for the PaaS logical components}
\label{fig:f11}
\end{figure}


\paragraph{}This figure illustrates a package design of the main PaaS services. We mainly distinguish :\newline
-	A control plane: which manages assets in the cluster, namely, nodes, pods and other api resources. \newline
-	An assortment of networking services which allow for ingress control in both the network and application layers.\newline
-	An authentication and authorization service: which is aimed to control access to the cluster.\newline
-	A distributed, scalable, and replicated storage backend which is independent of the infrastructure in place to provide data redundancy and disaster recovery.

\section*{Conclusion}
